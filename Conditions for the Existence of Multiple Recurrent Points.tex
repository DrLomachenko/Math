
Let $\mathcal{G}$ be a non-finite, compact metric space and $f: \mathcal{G} \to \mathcal{G}$ be a homeomorphism such that the metric of the images is not greater than the metric of the preimages. If there exist $r > 0$ and $x1, x2 \in \mathcal{G}$ such that for any $t \in \mathbb{N}$, $d(f^t(x1), f^t(x2)) \geq r$, then there exists more than one recurrent point.

Proof. Consider the sequences $\{ f^n(x1) \}$ and $\{ f^n(x2) \}$. Since $\mathcal{G}$ is compact, these sequences have convergent subsequences. From the condition on the metric, we get that these sequences do not have a common limit, and hence we can take $a, b \in \mathcal{G}$ such that $a \neq b$ and they are partial limits. Note that $\{ f^n(x1) \}$ and $\{ f^n(x2) \}$ may have the same partial limits, however they do not have the same common limit, and hence there will be at least two subsequences with different limits.

We will prove that $a$ is a recurrent point; the case with $b$ is analogous.

Suppose, on the contrary, that $a$ is not a recurrent point. Then for any $\epsilon > 0$, there exists $t \in \mathbb{N}$ such that $f^t(a) \notin B\epsilon(a)$, where $B\epsilon(a)$ is an $\epsilon$-neighborhood of $a$.

Since $f$ is a homeomorphism, $f^t$ is also a homeomorphism. Therefore, $f^t(B\epsilon(a))$ is an open set in $\mathcal{G}$.

On the other hand, for any $\delta$-neighborhood of $a$, by construction, there exists $x \in B\delta(a)$ such that $f^t(x) \in B\delta(a)$.

Take $\delta = \epsilon/8$. Then there exists $x \in B{\epsilon/8}(a)$ such that $f^t(x) \in B{\epsilon/8}(a)$.

Now consider an $\epsilon/4$-neighborhood of $a$. Since $f^t$ is a homeomorphism, $f^t(B{\epsilon/4}(a))$ is an open set. Moreover, by the condition on the metric, the diameter of $f^t(B{\epsilon/4}(a))$ is not greater than the diameter of $B{\epsilon/4}(a)$, which is $\epsilon/2$.

Since $f^t(x) \in B{\epsilon/8}(a)$, and the diameter, which we define as $\mathrm{diam}(S) = \sup{x, y \in S} d(x, y)$, of $f^t(B{\epsilon/4}(a))$ is not greater than $\epsilon/2$, we get $f^t(B{\epsilon/4}(a)) \subseteq B\epsilon(a)$.

Therefore, $f^t(a) \in B\epsilon(a)$, which contradicts our assumption.

Thus, our assumption that $a$ is not a recurrent point is false, and hence $a$ is a recurrent point.


